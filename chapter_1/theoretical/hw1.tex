\documentclass[UTF8]{ctexart}

\usepackage{amsmath, geometry, amssymb, framed, amsthm}
\geometry{left=2cm, right=2cm, top=2cm, bottom=2cm}

\newtheorem{theorem}{定理}%[subsection]
\newtheorem{definition}{定义}%[subsection]
\newtheorem{lemma}{引理}%[subsection]
\newtheorem{corollary}{推论}%[subsection]
\newtheorem{example}{例}%[subsection]
\newtheorem{proposition}{命题}%[subsection]
\newtheorem{remark}{注记}%[subsection]
\newtheorem{axiom}{公理}%[section]

\title{\vspace{-2cm}Homework for Chapter 1 (Theoretical Questions)}
\author{王笑同 \quad 3210105450 \quad 数学与应用数学(强基计划)2101}
\date{\today}

\linespread{1.9}

\begin{document}

\pagestyle{plain}

\maketitle

\textbf{Problem I. \& Problem II. }

These two problems can be extended to general cases. 

Suppose we are working on interval $[a_0,b_0]$, with initial length $l_0:=b_0-a_0$. By definition of bisection method, The length of interval at the $n$th step is
\[l_n=\dfrac{b_0-a_0}{2^n},\]

and $\dfrac{l_n}{2}=\dfrac{b_0-a_0}{2^{n+1}}$ is the supremum of the distance between the root $r$ and the midpoint of the interval.

Take $[a_0,b_0]=[1.5, 3.5]$, we have $l_n=2^{1-n}$, $\dfrac{l_n}{2}=2^{-n}$.

Let $\alpha$ be the root. To ensure the \textbf{relative} error $\dfrac{l_n}{2}\bigg / \alpha$ no more than $\varepsilon$, we need to have
\[\dfrac{l_n}{2\alpha}\leqslant\dfrac{l_n}{2a_0}=\dfrac{b_0-a_0}{2^{n+1}a_0}<\varepsilon,\]

which is equivalent to
\[n\geqslant \dfrac{\log(b_0-a_0)-\log\varepsilon-\log a_0}{\log 2}.\]

\quad

\textbf{Problem III.}

Since $p'(x)=12x^2-4x$, with the iteration formula $x_{n+1}=x_n-\dfrac{p(x_n)}{p'(x_n)}$, we have
\[\begin{aligned}
    x_1&=x_0-\dfrac{p(x_0)}{p'(x_0)}=-1-\dfrac{-3}{16}=-\dfrac{21}{16}=-0.8125,\\
    x_2&=x_1-\dfrac{p(x_1)}{p'(x_1)}\approx -0.7708,\\
    x_3&=x_2-\dfrac{p(x_2)}{p'(x_2)}\approx -0.7688,\\
    x_4&=x_3-\dfrac{p(x_3)}{p'(x_3)}\approx -0.7688.
\end{aligned}\]

\newpage

\textbf{Problem IV.}

Let the root be $\alpha$, and $e_n=x_n-\alpha$. Then there is $\xi\in[x_n,\alpha]$ such that
\[0=f(x_n)+(\alpha-x_n)f'(\xi).\]

That is, $f(x_n)=f'(\xi)e_n$. Then
\[e_{n+1}=x_{n+1}-\alpha=x_n-\dfrac{f(x_n)}{f'(x_n)}-\alpha=e_n-\dfrac{f(x_n)}{f'(x_n)}=e_n\left(1-\dfrac{f'(\xi_n)}{f'(x_0)}\right).\]

By comparing with $e_{n+1}=Ce_n^s$, we have
\[s=1,\qquad C=1-\dfrac{f'(\xi)}{f'(x_0)}.\]

\quad

\textbf{Problem V.}

The interation clearly converges when $x_0=0$. And, if $|x_0|<\dfrac\pi2$, then $|x_n|=|\arctan x_{n-1}|<\dfrac\pi2$, so it is a bounded sequence.

For $x_0\in\left(0,\dfrac\pi2\right)$, we have $x_{n+1}-x_n=\arctan x_n-x_n<0$, so the sequence decreases. By the monotone convergence theorem, the sequence converges. For $x_0\in\left(-\dfrac\pi2,0\right)$, we have similar results since $x<\arctan x$ when $x<0$.

\quad

\textbf{Problem VI.}

Write $x_1=\dfrac1p$, $x_{n+1}=\dfrac{1}{x_n+p}$. It suffices to show $x:=\lim\limits_{n\to\infty} x_n$ exists. 

By induction, it is easy to show that $x_n\in(0,1)$, $n\in\mathbb{Z}_+$.

Let $f(x)=\dfrac{1}{x+p}$, and then $f'(x)=-\dfrac{1}{(x+p)^2}$. It follows that
\[\lambda:=\max\limits_{x\in[0,1]}|f'(x)|<\dfrac{1}{p^2}<1.\]

By theorem 1.39, $f$ has a unique fixed point $\alpha$ on $[0,1]$, and the sequence $\{x_n\}$ converges to $\alpha$. From $\alpha=\dfrac{1}{\alpha+p}$, we see that $x=\alpha=\dfrac{-p+\sqrt{p^2+4}}{2}$.

\newpage

\textbf{Problem VII.}

Let $\alpha$ be the root in $[a_0,b_0]$. We have similar estimates as in Problem I and Problem II:
\[\text{Relative error\,}=\dfrac{|\alpha-l_n|}{|\alpha|}\leqslant\dfrac{l_n}{2^n|\alpha|}=\dfrac{b_0-a_0}{2^n|\alpha|}.\]

Assume the relative error can be controlled within $\varepsilon$, we have
\[n\geqslant\dfrac{\log(b_0-a_0)-\log\varepsilon-\log|\alpha|}{\log 2}-1.\]

However, $n\to\infty$ as $\varepsilon\to0^+$, leading to contradiction. Thus the relative error can't be appropriately measured.

\end{document}